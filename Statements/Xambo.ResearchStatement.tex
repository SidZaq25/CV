%------------------------------------
% Dario Taraborelli
% Typesetting your academic CV in LaTeX
%
% URL: http://nitens.org/taraborelli/cvtex
% DISCLAIMER: This template is provided for free and without any guarantee 
% that it will correctly compile on your system if you have a non-standard  
% configuration.
% Some rights reserved: http://creativecommons.org/licenses/by-sa/3.0/
%------------------------------------

%!TEX TS-program = xelatex
%!TEX encoding = UTF-8 Unicode

\documentclass[10pt, a4paper]{article}
\usepackage{fontspec} 

% DOCUMENT LAYOUT
\usepackage{geometry} 
\geometry{a4paper, textwidth=5.5in, textheight=8.5in, marginparsep=7pt, marginparwidth=.6in}
\setlength\parindent{0in}

% FONTS
\usepackage[usenames,dvipsnames]{xcolor}
\usepackage{xunicode}
\usepackage{xltxtra}
\defaultfontfeatures{Mapping=tex-text}
%\setromanfont [Ligatures={Common}, Numbers={OldStyle}, Variant=01]{Linux Libertine O}
%\setmonofont[Scale=0.8]{Monaco}
%%% modified by Karol Kozioł for ShareLaTeX use
\setmainfont[
  Ligatures={Common}, Numbers={OldStyle}, Variant=01,
  BoldFont=LinLibertine_RB.otf,
  ItalicFont=LinLibertine_RI.otf,
  BoldItalicFont=LinLibertine_RBI.otf
]{LinLibertine_R.otf}
\setmonofont[Scale=0.8]{DejaVuSansMono.ttf}

% ---- CUSTOM COMMANDS
\chardef\&="E050
\newcommand{\html}[1]{\href{#1}{\scriptsize\textsc{[html]}}}
\newcommand{\pdf}[1]{\href{#1}{\scriptsize\textsc{[pdf]}}}
\newcommand{\doi}[1]{\href{#1}{\scriptsize\textsc{[doi]}}}
% ---- MARGIN YEARS
\usepackage{marginnote}
\newcommand{\amper{}}{\chardef\amper="E0BD }
\newcommand{\years}[1]{\marginnote{\scriptsize #1}}
\renewcommand*{\raggedleftmarginnote}{}
\setlength{\marginparsep}{7pt}
\reversemarginpar

% HEADINGS
\usepackage{sectsty} 
\usepackage[normalem]{ulem} 
\sectionfont{\mdseries\upshape\Large}
\subsectionfont{\mdseries\scshape\normalsize} 
\subsubsectionfont{\mdseries\upshape\large} 

% PDF SETUP
% ---- FILL IN HERE THE DOC TITLE AND AUTHOR
\usepackage[%dvipdfm, 
bookmarks, colorlinks, breaklinks, 
% ---- FILL IN HERE THE TITLE AND AUTHOR
	pdftitle={Anna Xambó - vita},
	pdfauthor={Anna Xambó},
	pdfproducer={http://nitens.org/taraborelli/cvtex}
]{hyperref}  
\hypersetup{linkcolor=blue,citecolor=blue,filecolor=black,urlcolor=MidnightBlue} 

%\pagenumbering{gobble} % switch off page numbering

%HEADER & FOOTER
\pagenumbering{arabic}
\usepackage{lastpage}
\usepackage{fancyhdr}
\pagestyle{fancy}
\renewcommand{\headrulewidth}{0pt}
\lfoot{Anna Xambó, PhD}
\cfoot{Research Statement}%removes pagination at the center of the footer
\rfoot{\thepage\ of \pageref{LastPage}}

% DOCUMENT
\begin{document}
{\LARGE Research statement}\\[0.2cm]
Anna Xambó, PhD\\
\href{http://annaxambo.me}{annaxambo.me}

\section*{Vision \& Mission}

I envision pushing the boundaries of \textbf{technology}, \textbf{design}, and \textbf{experience} towards more collaborative, democratic and sustainable spaces, what I term tangible music computing. My mission is to do interdisciplinary research that embraces techniques and research methods from engineering, social sciences, and the arts for creating a new generation of interactive music systems.

\section*{Foci of research}

This research contributes to the fields of \textbf{HCI} and \textbf{sound} \& \textbf{music computing}.\\ 

My research has \textbf{three foci}:

\begin{enumerate}
\item \textbf{Technology}: using cutting-edge technology in real-time interactive musical systems and creative algorithms borrowed from MIR and machine learning that can be useful for real-time performance and musical improvisation e.g. live coding or algorithmic music.
\item \textbf{Design}: exploring novel aesthetics for real-time interactive musical systems e.g. tangible interfaces or wearable computing. 
\item \textbf{Experience}: bringing more democratic, collaborative and participatory experiences to the fore e.g. multichannel experiences, participatory performances, DIY workshops. 
\end{enumerate}


\section*{Related publications}

\subsection*{Technology}

\begin{itemize}
\item Xambó, A., Lerch, A., Freeman, J. (2016). Learning to code through MIR. In \emph{Extended abstracts for the Late-Breaking Demo Session of the 17th International Society for Music Information Retrieval Conference (ISMIR 2016)}. New York. 
\item Xambó, A., Freeman, J., Magerko, B., Shah, P. (2016). Challenges and new directions for collaborative live coding in the classroom. In \emph{ICLI 2016}. Brighton, UK.
\item Xambó, A. (2015). \emph{Tabletop Tangible Interfaces for Music Performance: Design and Evaluation}. Thesis. The Open University.
\item Xambó, A., Roma, G., Laney, R., Dobbyn, C. and Jordà, S. (2014). “SoundXY4: supporting tabletop collaboration and awareness with ambisonics spatialisation”. In Proceedings of the International Conference on New Interfaces for Musical Expression 2014 (NIME ’14). London. pp. 249–252.
\item Roma, G. and Xambó, A. (2008). “A tabletop waveform editor for live performance”. In Proceedings of the International Conference on New Interfaces for Musical Expression (NIME ’08). Genoa, Italy.
\item Xambó, A. (2008). \emph{Interfaces for Sketching Musical Compositions}. Unpublished master’s thesis. UPF.
\end{itemize}

\subsection*{Design}

\begin{itemize}
\item Xambó, A. (forthcoming), ``Embodied music interaction: creative design synergies between music performance and HCI''. In Price, S. and Broadhurst, S. eds. Digital Bodies: Creativity and Technology in the Arts and Humanities. Palgrave Macmillan, London.
\item Xambó, A. (2015). \emph{Tabletop Tangible Interfaces for Music Performance: Design and Evaluation}. Thesis. The Open University.
\item Xambó, A., Jewitt, C., and Price, S. (2014). “Towards an integrated methodological framework for understanding embodiment in HCI”. In Proceedings of the Extended Abstracts on Human Factors in Computing Systems (CHI ’14). Toronto. pp. 1411–1416.
\item Xambó, A., Roma, G., Laney, R., Dobbyn, C. and Jordà, S. (2014). “SoundXY4: supporting tabletop collaboration and awareness with ambisonics spatialisation”. In Proceedings of the International Conference on New Interfaces for Musical Expression 2014 (NIME ’14). London. pp. 249–252.
\item Xambó, A., Laney, R., Dobbyn, C. and Jordà, S. (2011). “Multi-touch interaction principles for collaborative real-time music activities: towards a pattern language”. In Proceedings of the International Computer Music Conference (ICMC ’11). Huddersffeld, UK. pp. 403–406.
\item Roma, G. and Xambó, A. (2008). “A tabletop waveform editor for live performance”. In Proceedings of the International Conference on New Interfaces for Musical Expression (NIME ’08). Genoa, Italy.
\end{itemize}

\subsection*{Experience}

\begin{itemize}
\item Xambó, A., Drozda, B., Weisling, A., Magerko, B., Huet, M., Gasque, T., Freeman, J. (2017) “Experience and ownership with a tangible computational music installation for informal learning”. In Proceedings of the Tangible, Embedded, and Embodied Interaction Conference (TEI ’17). Yokohama, Japan.
\item Bogdanov, D., Haro, M., Fuhrmann, F., Xambó, A., Gómez, E. and Herrera, P. (2013). Semantic audio content-based music recommendation and visualization based on user preference examples. \emph{Information Processing \& Management}, 49(1), pp. 13-33.
\item Freeman, J., Magerko, B., Edwards, D., Moore, R., McKlin, T., Xambó, A. (2015). \emph{EarSketch: a STEAM approach to broadening participation in computer science principles}. In Proceedings of the IEEE Research in Equity and Sustained Participation in Engineering, Computing, and Technology (RESPECT '15). Charlotte, NC. pp. 109-110.
\item Xambó, A. (2015). \emph{Tabletop Tangible Interfaces for Music Performance: Design and Evaluation}. Thesis. The Open University.
\item Xambó, A., Roma, G., Laney, R., Dobbyn, C. and Jordà, S. (2014). SoundXY4: supporting tabletop collaboration and awareness with ambisonics spatialisation. In \emph{Proceedings of the International Conference on New Interfaces for Musical Expression 2014 (NIME '14)}. London. pp. 249–252.
\item Haro, M.; Xambó, A.; Fuhrmann, F.; Bogdanov, D.; Gómez, E. and Herrera, P. (2010). The Musical Avatar: a visualization of musical preferences by means of audio content description. In \emph{Proceedings of the 5th Audio Mostly Conference (AM '10)}. Piteå, Sweden.
\end{itemize}


%\item Xambó, A., Hornecker, E., Marshall, P., Jordà, S., Dobbyn, C. and Laney, R. (2016). Exploring social interaction with a tangible music interface. \emph{Interacting with Computers}.
%\item Xambó, A., Hornecker, E., Marshall, P., Jordà, S., Dobbyn, C. and Laney, R. (2013). \emph{Let's jam the Reactable: peer learning during musical improvisation with a tabletop tangible interface}. ACM Transactions on Computer-Human Interaction (TOCHI), 20(6), pp. 36:1–36:34.

%\vspace{1cm}
\vfill{}
%\hrulefill

\begin{center}
{\scriptsize  Anna Xambó •\- Curriculum Vitae •\- Last updated: \today\- •\- %original: Last updated: \today\- •\- 
% ---- PLEASE LEAVE THIS BACKLINK FOR ATTRIBUTION AS PER CC-LICENSE
Typeset in \href{http://nitens.org/taraborelli/cvtex}{
%\fontspec{Times New Roman}
\XeTeX }\\
% ---- FILL IN THE FULL URL TO YOUR CV HERE
\href{https://github.com/axambo/CV/tree/master/Statements}{https://github.com/axambo/CV/tree/master/Statements}}
\end{center}

\end{document}