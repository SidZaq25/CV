	%------------------------------------
% Dario Taraborelli
% Typesetting your academic CV in LaTeX
%
% URL: http://nitens.org/taraborelli/cvtex
% DISCLAIMER: This template is provided for free and without any guarantee 
% that it will correctly compile on your system if you have a non-standard  
% configuration.
% Some rights reserved: http://creativecommons.org/licenses/by-sa/3.0/
%------------------------------------

%!TEX TS-program = xelatex
%!TEX encoding = UTF-8 Unicode

\documentclass[10pt, a4paper]{article}
\usepackage{fontspec} 

% DOCUMENT LAYOUT
\usepackage{geometry} 
\geometry{a4paper, textwidth=5.5in, textheight=8.5in, marginparsep=8pt, marginparwidth=.7in}%original: marginparsep=7pt, marginparwidth=.6in
\setlength\parindent{0in}

% FONTS
\usepackage[usenames,dvipsnames]{xcolor}
\usepackage{xunicode}
\usepackage{xltxtra}
\defaultfontfeatures{Mapping=tex-text}
%\setromanfont [Ligatures={Common}, Numbers={OldStyle}, Variant=01]{Linux Libertine O}
%\setmonofont[Scale=0.8]{Monaco}
%%% modified by Karol Kozioł for ShareLaTeX use
\setmainfont[
  Ligatures={Common}, Numbers={OldStyle}, Variant=01,
  BoldFont=LinLibertine_RB.otf,
  ItalicFont=LinLibertine_RI.otf,
  BoldItalicFont=LinLibertine_RBI.otf
]{LinLibertine_R.otf}
\setmonofont[Scale=0.8]{DejaVuSansMono.ttf}

% ---- CUSTOM COMMANDS
\chardef\&="E050
\newcommand{\html}[1]{\href{#1}{\scriptsize\textsc{[html]}}}
\newcommand{\pdf}[1]{\href{#1}{\scriptsize\textsc{[pdf]}}}
\newcommand{\doi}[1]{\href{#1}{\scriptsize\textsc{[doi]}}}
% ---- MARGIN YEARS
\usepackage{marginnote}
\newcommand{\amper{}}{\chardef\amper="E0BD }
\newcommand{\years}[1]{\marginnote{\scriptsize #1}}
\renewcommand*{\raggedrightmarginnote}{} %\original: raggedleftmarginnote
\setlength{\marginparsep}{8pt}%original: 7pt
\reversemarginpar

% HEADINGS
\usepackage{sectsty} 
\usepackage[normalem]{ulem} 
\sectionfont{\mdseries\upshape\Large}
\subsectionfont{\mdseries\scshape\normalsize} 
\subsubsectionfont{\mdseries\upshape\large} 

% PDF SETUP
% ---- FILL IN HERE THE DOC TITLE AND AUTHOR
\usepackage[%dvipdfm, 
bookmarks, colorlinks, breaklinks, 
% ---- FILL IN HERE THE TITLE AND AUTHOR
	pdftitle={Anna Xambó - vita},
	pdfauthor={Anna Xambó},
	pdfproducer={http://nitens.org/taraborelli/cvtex}
]{hyperref}  
\hypersetup{linkcolor=blue,citecolor=blue,filecolor=black,urlcolor=MidnightBlue,linkcolor=MidnightBlue} 

% AVOID WIDOW LINES
\usepackage[all]{nowidow}

%HEADER & FOOTER
\usepackage{lastpage}
\usepackage{fancyhdr}
\pagestyle{fancy}
\renewcommand{\headrulewidth}{0pt}
\lfoot{Anna Xambó, PhD}
\cfoot{Curriculum Vitae}%removes pagination at the center of the footer
\rfoot{\thepage\ of \pageref{LastPage}}

% CUSTOM
\usepackage{microtype}
%\hyphenpenalty 10000
%\exhyphenpenalty 1000
%\widowpenalty 10000
%\clubpenalty 10000
%\interfootnotelinepenalty=10000


% DOCUMENT
\begin{document}
{\Huge Anna Xambó}\\[0.1cm]
\textsc{BA, MA, MSc, PhD}\\[0.9cm]
\emph{Associate Professor in Music Technology}\\
Department of Music\\
NTNU (Norwegian University of Science and Technology)\\
7491 Trondheim (Norway)\\[.2cm]
%\textsc{phone}: \texttt{(+44) (0) 737-879-6645}\\
\textsc{email}: \href{mailto:anna.xambo@ntnu.no}{anna.xambo@ntnu.no}\\
\textsc{webpage}: \href{http://annaxambo.me/}{annaxambo.me}\\ 

%%\hrule
\section*{Current Position}
\emph{Associate Professor in Music Technology}, Department of Music, Norwegian University of Science and Technology (NTNU).\\
\emph{Visiting Lecturer}, Centre for Digital Music (C4DM), School of Electronic Engineering and Computer Science (EECS), Queen Mary University of London (QMUL).

%%\hrule
\section*{Areas of Interest}
Design of Digital Musical Instruments (DMIs) • Real-time Interactive Systems for Music Performance • Human-Computer Interaction • Interaction Design • Tangible, Physical \& Social Computing • Computer-Supported Collaborative, Participatory \& Improvisation Music • Live Coding • Real-time Music Information Retrieval and Machine Learning • Multichannel Spatialization • Generative \& Algorithmic Music • Immersive Sound Experiences • Women in Music Tech • Arts \& Social Sciences Research Methods • STEAM Education • Data Visualization • Creative Programming

%\hrule
\section*{Education}
\noindent
\years{2015}\textsc{PhD}, The Open University (OU), UK \& \textsc{Dra.}, Universitat Pompeu Fabra (UPF), Spain.\\
Major: Music computing \& HCI.\\ 
Dissertation: \emph{Tabletop Tangible Interfaces for Music Performance: Design and Evaluation}.\\
\years{2008}\textsc{MSc} in Information, Communication and Audiovisual Media Technologies, UPF, Spain.\\
Major: Music computing \& HCI.\\ 
Dissertation: \emph{Interfaces for Sketching Musical Compositions}.\\
\years{1999}\textsc{Master} in Video, Animation and Multimedia Design, Media Art Institute Fak d'Art, Spain.\\
\years{1996}\textsc{BA, MA} in Social and Cultural Anthropology, Universitat de Barcelona (UB), Spain.

%\hrule
\section*{Dissertation}
\noindent
\years{Title}\textbf{Xambó, A.} (2015). \emph{Tabletop Tangible Interfaces for Music Performance: Design and Evaluation}.\\
\years{Advisors}Dr Robin Laney, Mr Chris Dobbyn and Prof Sergi Jordà.\\
\years{Examiners}Prof Eduardo Reck Miranda and Dr Janet van der Linden.\\
\years{Website}\href{http://oro.open.ac.uk/42473/}{http://oro.open.ac.uk/42473/}\\[.3cm]

%\hrule
\section*{Music Education}
\noindent
\subsection*{Classical Training}
\years{1983--1987}\textsc{Piano}, Conservatori Superior de Música del Liceu, Barcelona.\\
\years{1982--1988}\textsc{Music Theory \& Solfege}, Conservatori Superior de Música del Liceu, Barcelona.
\noindent
\subsection*{Workshops}
\years{Forthcoming}\textsc{Spatial Audio Workshop} by Eric Lyon. Virginia Tech, Blacksburg, Virginia, USA.\\
\years{2014}\textsc{Taller composición acusmática} (Acousmatic Composition Workshop) by Beatriz Ferreyra. Barcelona.\\
\years{2012}\textsc{Síntesi no estàndard: tècniques, estètiques, extensions} (Non-Standard Synthesis: Techniques, Aesthetics, Extensions) by Luc Döbereiner. Barcelona.\\
\years{2009}\textsc{Taller construeix el teu propi sintetitzador} (Build Your Own Synthesizer Workshop) by Tom Bugs. Barcelona.\\
\years{2008}\textsc{SMC Summer School} by Xavier Serra, Marc Leman, Benjamin Knapp, and the Casa Paganini - InfoMus Lab. Genoa, Italy.\\
\years{2006}\textsc{El món com a instrument} (The world as an instrument) by Francisco López. Barcelona.\\
\years{1998}\textsc{Improvització mètode Cobra} (Cobra Improvisation Method) by Orquestra del Caos. Barcelona.

%\hrule
\section*{Teaching Education}
\noindent
\years{2018--present}\textsc{Learning and Teaching in Higher Education}. Instructors: Emma Kennedy, Alison Gilmour, Maren Thom. QMUL. London.\\
\years{2018}\textsc{Women into Leadership}. Instructor: Lorraine Smith. QMUL. London.\\
\years{2017}\textsc{Communication Skills for Teaching for International Faculty, Postdocs, and Visiting Scholars}. Instructor: Katherine Samford. Georgia Institute of Technology (Georgia Tech). Atlanta, GA, USA.

%%\hrule
\section*{Employment}
\noindent
\years{08/2018--present}\textsc{Associate Professor in Music Technology}. Department of Music, NTNU. \\
\years{08/2018--present}\textsc{Visiting Lecturer}. C4DM, School of EECS, QMUL. \\
\years{10/2017--07/2018}\textsc{Postdoctoral Research Assistant}. C4DM, School of EECS, QMUL. \\
\years{07/2015--09/2017}\textsc{Postdoctoral Fellow}. Center for Music Technology (GTCMT) | Digital Media Program, Georgia Tech. \\
\years{08/2013--09/2014}\textsc{Research Fellow}. London Knowledge Lab, UCL Institute of Education. London. \\
%{\small Qualitative data collection and analysis (6 sites), development of research tools and processes, dissemination activities and paper writing of results.}\\
\years{02/2004--06/2010}\textsc{Co-Founder, Project Manager, Web Designer \& Web Developer}. Nodular Soft. Barcelona. \\
%{\small Freelance studio focused on user-centric software and AV communication, development of community websites using several CMS, development of AV programs under specific needs, and usability consultancy.} \\
\years{01/2008--07/2009}\textsc{Web Designer \&  Web Developer Project Officer}. Music Technology Group, UPF. Barcelona. \\
%{\small Web design and web development of the web 2.0 Sons de Barcelona (barcelona.freesound.org). Web design of the corporate portal of the research group MTG (mtg.upf.edu). Graphic design of the corporate brochure of the MTG (courses 2008--2009 and 2009--2010).}\\
\years{11/2007--06/2009}\textsc{Web Designer \&  Web Developer Project Officer}. Uaalah!!. Barcelona. \\
%{\small Design and programming of a self-manageable interactive catalogue for CD-ROM about Bürkert products. Flash programming of the online tea shop of Sans \& Sans (sansisans-finetea.com).}\\
\years{08/2005--09/2006}\textsc{Web Designer \&  Motion Graphic Designer}. CCRTVi | TV3 Interactiva. Sant Just Desvern, Barcelona. \\
%{\small Web interface design of different portals of the Catalan TV corporation (tv3.cat, catradio.cat, ritmes.net, among others). Web interface design of the sitcom Lo Cartanyà (locartanya.com). Mobile design of the prototype 3alacarta.}\\
\years{05/2001--08/2002}\textsc{Web Designer \&  Motion Graphic Designer}. TerraNetworks | UranoFilms. Barcelona. \\
%{\small Web interface design and flash design of AV internet content about electronic music and digital culture.}\\
\years{04/2000--05/2001}\textsc{Web Designer \&  Motion Graphic Designer}. MediaPark | ParkNet, Barcelona. \\[1.3cm]%hack
%{\small Flash design of animations games for the Internet soccer portal futvol.com.} \\[1.1cm]

%\hrule
\section*{Honors \& Awards}
\noindent

\subsection*{Research Grants, Honors \& Awards}
\years{2018--2021} \textsc{Startup grant (Startpakke)} (250,000 NOK), awarded to raise the number of women in male dominated areas. NTNU, Trondheim, Norway.\\
\years{2017} NCWIT Engagement Excellence Award (\$5,000 cash award) to Greg Hendler, Léa Ikkache, Brandon Westergaard, Anna Xambó, Doug Edwards, Brian Magerko, and Jason Freeman (Earsketch), Georgia Tech.\\
\years{2016}\textsc{Women in Music Information Retrieval (WiMIR) grant}, awarded to attend the ISMIR 2016 conference. New York University (NYU), New York.\\
\years{10/2010--07/2013}\textsc{Fully-Funded Full-Time OU PhD scholarship}. The Open University, Milton Keynes, UK.\\
\years{03/2010--06/2010}\textsc{Fully-Funded OU Visiting Research Studentship}. The Open University, Milton Keynes, UK.

\subsection*{Artistic Grants, Honors \& Awards}
\years{05/2004}\textsc{First prize award Minima Festival}. Gandía, Spain. \\
Category: Experimental Video. \\
Project: ``Cosmogonias". \\
Role: Creator \& Director.

%\hrule
\section*{Grants \& Funding}
\noindent

\subsection*{Principal Investigator}

\years{11/2003-10/2004}\textsc{Teaching innovation project grant} \\
Funding body: Fundació Caixa de Sabadell. \\
Project: ``Crossmedia infantil: Estudio sobre las nuevas tecnologías y la comunicación audiovisual en la escuela infantil y primaria (Crossmedia for Children: New Technologies and Audiovisual Communication in Primary Education)''.\\
Role: PI. \\
Collaborators: Eladi Martos (Co-PI), UB. \\
Total Dollar Amount: \$3,300\\
Candidate’s Share: 50\% (\$1,650)

\subsection*{Collaborator}

\years{09/2016-08/2020}\textsc{Advancing Informal STEM Learning Grant} \\
Funding body: National Science Foundation (NSF). \\
Project: ``Collaborative Research: Mixing Learning Experiences for Computer Programming Across Museums, Classrooms, and the Home Using Computational Music''. Award Number: 1612644. \\
Organization: Georgia Tech Research Corporation. \\
Role: Postdoctoral Fellow and Co-Writer of the grant proposal. \\
Collaborators: Brian Magerko (PI), Jason Freeman (Co-PI), Mike Horn (Co-PI).\\
Total Dollar Amount: \$2,517,690.00

\subsection*{Fundraiser}

\years{2017--2018}\textsc{Female Laptop Orchestra + Women in Music Tech seminar and concert at SARC} \\
Role: Co-organizer and participant of the event. \\
Total Fundraised British Pound Amount: \textsterling2,000\\[.1cm]
\years{05/2016--05/2017}\textsc{Women in Music Tech} \\
Role: Co-Founder \& Co-Chair of the organization. \\
Total Fundraised Dollar Amount: \$11,450\\[.1cm]
\years{08/2016--05/2017}\textbf{2016-2017 Academic Year}\\
Funding body: School of Music, Georgia Tech. \\
Total Dollar Amount: \$2,500\\[.1cm]
\years{08/2016--11/2016}\textbf{Fall 2016 Concert Event}\\
Funding body: College of Design, Georgia Tech. \\
Total Dollar Amount: \$2,000\\[.1cm]
Funding body: ADVANCE program, Georgia Tech. \\
Total Dollar Amount: \$1,000\\[.1cm]
Funding body: Women's Resource Center, Georgia Tech. \\
Total Dollar Amount: \$250\\[.1cm]
\years{01/2017--05/2017}\textbf{Spring 2017 Actions}\\
Funding body: School of Music, Georgia Tech. \\
Total Dollar Amount: \$2,400\\[.1cm]
Funding body: College of Design Council Diversity, Georgia Tech. \\
Total Dollar Amount: \$1,500\\[.1cm]
Funding body: ADVANCE Program, Georgia Tech. \\
Total Dollar Amount: \$1,000\\[.1cm]
Funding body: Women's Resource Center, Georgia Tech. \\
Total Dollar Amount: \$500\\[.1cm]
Funding body: Digital Media Program, School of Literature, Media, and Communication, Georgia Tech. \\
Total Dollar Amount: \$300

\subsection*{Creator | Director}

\years{09/2001--08/2002}\textsc{Audiovisual production grant} \\
Funding body: Departament de Cultura de la Generalitat de Catalunya (Department of Culture of Catalan Government).\\
Project: ``Transdata Pr.''.  \\
Role: Creator, Video Editor \& Director.\\
Collaborators: Gerard Roma (music), Oscar Abril Ascaso (essay). \\
Total Dollar Amount: \$3,300 \\ 
Candidate’s Share: 50\% (\$1,650)\\

\years{09/1998--08/1999}\textsc{Audiovisual production grant} \\
Funding body: Departament de Cultura de la Generalitat de Catalunya (Department of Culture of Catalan Government).\\
Project: ``Mitösömä''.  \\
Role: Creator, Animation Editor \& Director.\\
Collaborators: Gerard Roma (music). \\
Grant Amount: 3,000€ (\$3,335) \\
Candidate’s Share: 50\% (\$1,650)



\section*{Research Profiles}
\noindent

\textbullet \- \href{https://scholar.google.com/citations?user=yi3WXM8AAAAJ}{Scholar Google}\\
\textbullet \- \href{http://oro.open.ac.uk/view/person/ax22.html}{Open Research Online}\\
\textbullet \- \href{http://open.academia.edu/AnnaXambo}{Academia.edu}\\
\textbullet \- \href{http://www.researchgate.net/profile/Anna_Xambo}{ResearchGate}

\section*{Publications}
\noindent

\subsection*{Books}
\noindent
\years{2004}\textbf{Xambó, A.} (2004). \emph{Herramientas De Diseño Digital / Digital Design Tools}. Madrid: Anaya-Multimedia. ISBN 8441516979.

\subsection*{Peer-Reviewed Book Chapters}
\noindent
\years{2016}\textbf{Xambó, A.} (2017), “Embodied Music Interaction: Creative Design Synergies Between Music Performance and HCI". In Price, S. and Broadhurst, S. eds. Digital Bodies: Creativity and Technology in the Arts and Humanities. Palgrave Macmillan, London. pp. 207--220. ISBN 9781349952410.\\
\years{2013}\textbf{Xambó, A.}, Laney, R., Dobbyn, C. and Jordà, S. (2013). “Video Analysis for Evaluating Music Interaction: Musical Tabletops". In Holland, S., Wilkie, K., Mulholland, P. and Seago, A. eds. Music and Human-Computer Interaction. Springer, London. pp. 241--258. ISBN 9781447129905.

\subsection*{Journal Articles}
\noindent
\years{2018b}Roma, G. and \textbf{Xambó, A.} and Freeman, J. (2018). “User-independent Accelerometer Gesture Recognition for Participatory Mobile Music". \emph{Journal of Audio Engineering Society}, 66(6), pp. 430--438.\\
\years{2018a}\textbf{Xambó, A.}, Roma, G., Shah, P., Tsuchiya, T., Freeman, J. and Magerko, B. (2018). “Turn-taking and Online Chatting in Co-located and Remote Collaborative Music Live Coding". \emph{Journal of Audio Engineering Society}, 66(4), pp. 253--256.\\
\years{2017c}\textbf{Xambó, A.}, Hornecker, E., Marshall, P., Jordà, S., Dobbyn, C. and Laney, R. (2017). “Exploring Social Interaction with a Tangible Music Interface". \emph{Interacting with Computers}, 29(2), pp. 248--270.\\
\years{2017b}Jewitt, C., Price, S., \textbf{Xambó, A.} (2017). “Conceptualising and Researching the Body in Digital Contexts: Towards New Methodological Conversations Across the Arts and Social Sciences". \emph{Qualitative Research}, 17(1), pp. 37--53.\\
\years{2017a}Jewitt, C., \textbf{Xambó, A.} and Price, S. (2017). “Exploring Methodological Innovation in the Social Sciences: The Body in Digital Environments and the Arts". \emph{International Journal of Social Research Methodology},  20(1), pp. 105--120.\\
\years{2013b}\textbf{Xambó, A.}, Hornecker, E., Marshall, P., Jordà, S., Dobbyn, C. and Laney, R. (2013). “Let's Jam the Reactable: Peer Learning during Musical Improvisation with a Tabletop Tangible Interface". \emph{ACM Transactions on Computer-Human Interaction}, 20(6), pp. 36:1--36:34.\\
\years{2013a}Bogdanov, D., Haro, M., Fuhrmann, F., \textbf{Xambó, A.}, Gómez, E. and Herrera, P. (2013). “Semantic Audio Content-based Music Recommendation and Visualization based on User Preference Examples". \emph{Information Processing \& Management}, 49(1), pp. 13--33.

\subsection*{Peer-Reviewed Conference Papers}
\noindent

\years{2018f}Roma, G., \textbf{Xambó, A.}, Green, O., Tremblay, P.A. (2018) "A Javascript Library for Flexible Visualization of Audio Descriptors". In Proceedings of the Web Audio Conference (WAC '18). Berlin, Germany.\\
\years{2018e}Pauwels, J., \textbf{Xambó, A.}, Roma, G., Barthet, M. Fazekas, G. (2018) "Exploring Real-time Visualisations to Support Chord Learning with a Large Music Collection". In Proceedings of the Web Audio Conference (WAC '18). Berlin, Germany.\\
\years{2018d}\textbf{Xambó, A.}, Pauwels, J., Roma, G., Barthet, M. Fazekas, G. (2018) "Jam with Jamendo: Querying a Large Music Collection by Chords from a Learner’s Perspective". In Proceedings of Audio Mostly 2018: Sound in Immersion and Emotion (AM '18). Wrexham, United Kingdom. \\
\years{2018c}\textbf{Xambó, A.} (2018) “Who Are the Women Authors in NIME?---Improving Gender Balance in NIME Research". In \emph{Proceedings of the New Interfaces for Musical Expression (NIME '18)}. Blacksburg, Virginia, USA, pp. 174--177.\\ 
\years{2018b}\textbf{Xambó, A.}, Roma, G., Lerch, A., Barthet, M., Fakekas, G. (2018) “Live Repurposing of Sounds: MIR Explorations with Personal and Crowdsourced Databases". In \emph{Proceedings of the New Interfaces for Musical Expression (NIME '18)}. Blacksburg, Virginia, USA, pp. 364--369.\\ 
\years{2018a}Weisling, A., \textbf{Xambó, A.}, Olowe, I., Barthet, M. (2018) “Surveying the Compositional and Performance Practices of Audiovisual Practitioners". In \emph{Proceedings of the New Interfaces for Musical Expression (NIME '18)}. Blacksburg, Virginia, USA, pp. 344--345.\\ 
\years{2017d}\textbf{Xambó, A.}, Shah, P., Roma, G., Freeman, J., Magerko, B. (2017) “Turn-taking and Chatting in Collaborative Music Live Coding". In \emph{Proceedings of the Audio Mostly Conference (AM '17)}. London.\\ 
\years{2017c}Roma, G., \textbf{Xambó, A.}, Freeman, J. (2017) “Handwaving: Gesture Recognition for Participatory Mobile Music". In \emph{Proceedings of the Audio Mostly Conference (AM '17)}. London.\\
\years{2017b}Roma, G., \textbf{Xambó, A.}, Freeman, J. (2017) “Loop-aware Audio Recording for the Web". In \emph{Proceedings of the Web Audio Conference 2017 (WAC '17)}. London\\ 
\years{2017a}\textbf{Xambó, A.}, Drozda, B., Weisling, A., Magerko, B., Huet, M., Gasque, T., Freeman, J. (2017) Experience and Ownership with a Tangible Computational Music Installation for Informal Learning. In \emph{Proceedings of the Tangible, Embedded, and Embodied Interaction Conference (TEI '17)}. Yokohama, Japan. pp. 351--360.\\ 
\years{2016b}Freeman, J., Magerko, B., Edwards, D., Miller, M., Moore, R., \textbf{Xambó, A.} (2016). “Using EarSketch to Broaden Participation in Computing and Music". In \emph{Proceedings of the 13th Sound and Music Computing Conference (SMC 2016)}. Hamburg, Germany. pp. 156--163.\\
\years{2016a}\textbf{Xambó, A.}, Freeman, J., Magerko, B., Shah, P. (2016). “Challenges and New Directions for Collaborative Live Coding in the Classroom". In \emph{Proceedings of the International Conference of Live Interfaces (ICLI 2016)}. Brighton, UK.\\
\years{2014}\textbf{Xambó, A.}, Roma, G., Laney, R., Dobbyn, C. and Jordà, S. (2014). “SoundXY4: Supporting Tabletop Collaboration and Awareness with Ambisonics Spatialisation". In \emph{Proceedings of the International Conference on New Interfaces for Musical Expression 2014 (NIME '14)}. London. pp. 249--252.\\
\years{2013}Bogdanov, D., Haro, M., Fuhrmann, F., \textbf{Xambó, A.}, Gómez, E. and Herrera, P. (2013). “A Content-based System for Music Recommendation and Visualization of User Preferences Working on Semantic Notions". In \emph{IEEE 9th International Workshop on Content-Based Multimedia Indexing (CBMI '13)}. Madrid. pp. 249--252.\\
\years{2012}Roma, G., \textbf{Xambó, A.}, Herrera, P. and Laney, R. (2012). “Factors in human recognition of timbre lexicons generated by data clustering". In \emph{Proceedings of the 9th Sound and Music Computing Conference (SMC 2012)}. Copenhagen, Denmark. pp. 23--30.\\
\years{2011c}\textbf{Xambó, A.}, Laney, R., Dobbyn, C. and Jordà, S. (2011). “Multi-touch Interaction Principles for Collaborative Real-time Music Activities: Towards a Pattern Language". In \emph{Proceedings of the International Computer Music Conference (ICMC '11)}. Huddersfield, UK. pp. 403--406.\\
\years{2011b}\textbf{Xambó, A.}, Laney, R. and Dobbyn, C. (2011). “TOUCHtr4ck: Democratic Collaborative Music". In \emph{Proceedings of the Tangible, Embedded, and Embodied Interaction Conference (TEI '11)}. Funchal, Madeira. pp. 309--312.\\
\years{2011a}Milne, A. J., \textbf{Xambó, A.}, Laney, R., Sharp, D. B., Prechtl, A. and Holland, S. (2011). “Hex Player — A Virtual Musical Controller". In \emph{Proceedings of the International Conference on New Interfaces for Musical Expression (NIME '11)}. Oslo, Norway. pp. 244--247.\\
\years{2010b}Laney, R., Dobbyn, C., \textbf{Xambó, A.}, Schirosa, M., Miell, D., Littleton, K. and Dalton, N. (2010). “Issues and Techniques for Collaborative Music Making on Multi-touch Surfaces". In \emph{Proceedings of the 7th Sound and Music Computing Conference (SMC 2010)}. Barcelona. pp. 146–153.\\
\years{2010a}Haro, M., \textbf{Xambó, A.}, Fuhrmann, F., Bogdanov, D., Gómez, E. and Herrera, P. (2010). “The Musical Avatar: A Visualization of Musical Preferences by means of Audio Content Description". In \emph{Proceedings of the 5th Audio Mostly Conference (AM '10)}. Piteå, Sweden.\\
\years{2008}Roma, G. and \textbf{Xambó, A.} (2008). “A Tabletop Waveform Editor for Live Performance". In \emph{Proceedings of the International Conference on New Interfaces for Musical Expression (NIME '08)}. Genoa, Italy.

\subsection*{Peer-Reviewed Abstracts with Proceedings}
\noindent
\years{2018b}Skach, S., \textbf{Xambó, A.}, Turchet, L., Stolfi, A., Stewart, B., Barthet, M. (2018). “Embodied Interactions with E-Textiles and the Internet of Sounds for Performing Arts". In \emph{Proceedings of the Twelfth International Conference on Tangible, Embedded, and Embodied Interaction (TEI '18)}. Stockholm, Sweden. pp. 80--87.\\
\years{2018a}Weisling, A., \textbf{Xambó, A.} (2018). “Beacon: Exploring Physicality in Digital Performance". In \emph{Proceedings of the Twelfth International Conference on Tangible, Embedded, and Embodied Interaction (TEI '18)}. Stockholm, Sweden. pp. 586--591.\\
\years{2017a}\textbf{Xambó, A.}, Roma, G. (2017). “Hyperconnected Action Painting". In \emph{Proceedings of the Web Audio Conference 2017 (WAC '17)}. London.\\
\years{2016c}Tsuchiya, T., \textbf{Xambó, A.}, Freeman, J. (2016). “Adapting DAW-driven Musical Language to Live Coding: A Case Study in EarSketch". In \emph{Late-Breaking Demo of the Second International Conference on Live Coding (ICLC '16)}. Hamilton, Canada.\\ 
\years{2016b}\textbf{Xambó, A.}, Lerch, A., Freeman, J. (2016). “Learning to Code Through MIR". In \emph{Extended abstracts for the Late-Breaking Demo Session of the 17th International Society for Music Information Retrieval Conference (ISMIR 2016)}. New York.\\
\years{2016a}Roma, G., \textbf{Xambó, A.}, Freeman, J. (2016). “Do the Buzzer Shake". In \emph{International Conference of Live Interfaces (ICLI 2016)}. Brighton, UK.\\
\years{2015}Freeman, J., Magerko, B., Edwards, D., Moore, R., McKlin, T., \textbf{Xambó, A.} (2015). “EarSketch: A STEAM Approach to Broadening Participation in Computer Science Principles". In \emph{Proceedings of the IEEE Research in Equity and Sustained Participation in Engineering, Computing, and Technology (RESPECT '15)}. Charlotte, NC. pp. 109--110.\\
\years{2014}\textbf{Xambó, A.}, Jewitt, C., and Price, S. (2014). “Towards an Integrated Methodological Framework for Understanding Embodiment in HCI". In \emph{Proceedings of the Extended Abstracts on Human Factors in Computing Systems (CHI '14)}. Toronto. pp. 1411--1416.

\subsection*{Position \& Workshop Papers}
\noindent

\years{2017}\textbf{Xambó, A.}, Roma, G., Shah, P., Freeman, J., Magerko, B. (2017) “Computational Challenges of Co-creation in Collaborative Music Live Coding: An Outline". 2017 Co-Creation Workshop at the International Conference on Computational Creativity. Atlanta, GA, USA.\\ 
\years{2012}\textbf{Xambó, A.}; Laney, R.; Dobbyn, C. and Jordà, S. (September 11, 2012). “Towards a Taxonomy for Video Analysis on Collaborative Musical Tabletops". In \emph{BCS HCI 2012 Workshop on video analysis techniques for HCI}. Birmingham, UK.\\
\years{2011}\textbf{Xambó, A.}; Laney, R.; Dobbyn, C. and Jordà, S. (July 4, 2011). ``Collaborative Music Interaction on Tabletops: An HCI Approach'''. In \emph{BCS HCI 2011 Workshop on When Words Fail: What can Music Interaction tell us about HCI?}. Newcastle Upon Tyne.

\subsection*{Reports \& Working Papers}
\noindent

\years{2008}\textbf{Xambó, A.} (2008). Interfaces for Sketching Musical Compositions. Unpublished master's thesis. UPF.\\ 
\years{2004}\textbf{Xambó, A.}, Martos, E. (2004). Crossmedia Infantil: Estudi sobre les noves tecnologies i la comunicació audiovisual a l'escola infantil i primària (Report of New Technologies and Audiovisual Communication in the Primary Education). Unpublished report. Supported by Fundació Caixa de Sabadell. In collaboration with UB.

\section*{Talks, Panels \& Oral Presentations}
\noindent

\subsection*{External}
\noindent

\years{2018e} \textsc{Oral Presenter} together with Pauwels, J.  (September 19, 2018). "Exploring Real-time Visualisations to Support Chord Learning with a Large Music Collection". \emph{WAC '18}. \emph{Technische Universität Berlin}, Berlin, Germany.\\
\years{2018d} \textsc{Oral Presenter} together with Pauwels, J. (September 14, 2018). "Jam with Jamendo: Querying a Large Music Collection by Chords from a Learner’s Perspective". \emph{AM '18}. \emph{University of Wrexham}, Wrexham, UK.\\
\years{2018c} \textsc{Oral Presenter}. (July 12, 2018). “Audio Commons: Challenges and Opportunities of Using Online Repositories in Music Production and Performance". \emph{Filmuniversität Babelsberg Konrad Wolf}. Potsdam, Germany.\\
\years{2018b} \textsc{Oral Presenter}. (June 6, 2018). “Who Are the Women Authors in NIME?---Improving Gender Balance in NIME Research". \emph{NIME '18}. Virginia Tech, Blacksburg, Virginia, USA.\\
\years{2018a} \textsc{Oral Presenter}. (April 21, 2018). “Live Repurposing of Crowdsourced Sounds: Challenges and Opportunities of Using Online Repositories in Music Performance". \emph{Sonorities Symposium, Sonorities Festival}. Queen's University Belfast, Belfast, Northern Ireland.\\
\years{2017} \textsc{Oral Presenter}. (August 24, 2017). “Turn-taking and Chatting in Collaborative Music Live Coding". \emph{AM '17}. London.\\
\years{2016b} \textsc{Oral Presenter}. (July 2, 2016). “Challenges and New Directions for Collaborative Live Coding in the Classroom". \emph{ICLI 2016}. University of Sussex, Brighton, UK.\\
\years{2016a} \textsc{Keynote Speaker}. (April 22, 2016). “Anna Xambó and Liz Dobson in Conversation". \emph{Women in Sound Women on Sound 2016: Educating girls in sound} at University of Lancaster. Lancaster, UK.\\
\years{2015} \textsc{Lightning Talk Speaker}. (August 14, 2015). “EarSketch: A STEAM Approach to Broadening Participation in Computer Science Principles". \emph{RESPECT 2015}. Charlotte, NC. USA.\\
\years{2014b} \textsc{Oral Presenter}. (July 1, 2014). “SoundXY4: Supporting Tabletop Collaboration and Awareness with Ambisonics Spatialisation". \emph{NIME '14}. Goldsmiths University, London.\\
\years{2014a} \textsc{Oral Presenter}. (April 30, 2014). “Let's Jam the Reactable: Peer Learning during Musical Improvisation with a Tabletop Tangible Interface". \emph{CHI '14}. Toronto, ON, Canada.\\
\years{2013} \textsc{Oral Presenter}. (November 11, 2013). “Tabletop Tangible Interfaces for Music Performance and Implications for Tabletop Research". \emph{School of Computing}, University of Kent. Kent, UK.\\
\years{2011b} \textsc{Oral Presenter}. (August 2, 2011). “Multi-touch Interaction Principles for Collaborative Real-time Music Activities: Towards a Pattern Language". \emph{ICMC '11}. University of Huddersfield. Huddersfield, UK.\\
\years{2011a} \textsc{Oral Presenter}. (July 4, 2011). “Collaborative Music Interaction on Tabletops: An HCI Approach". \emph{BCS HCI 2011 Workshop on When Words Fail: What can Music Interaction tell us about HCI?}. Newcastle Upon Tyne, UK.\\
\years{2010} \textsc{Oral Presenter}. (July 23, 2010). “Issues and Techniques for Collaborative Music Making on Multi-touch Surfaces". \emph{SMC '10}. Universitat Pompeu Fabra, Barcelona.\\
\years{2008c} \textsc{Panel Member} together with Alsina, A., Ferrete, J. and Roma, G. (October 31, 2008). “Freesound, Sons de Barcelona y Freesound Radio: Proyectos colaborativos alrededor del sonido" (Freesound, Sons de Barcelona \& Freesound Radio: Collaborative Projects around sound). \emph{IV Cicle de Converses d'Antropologia Sonora}, Institució Milá i Fontanals (CSIC). Barcelona.\\
\years{2008b} \textsc{Panel Member} together with Alsina, A., Ferrete, J. and Roma, G. (2008). “Freesound.org, Freesound Radio i Sons de Barcelona" (Freesound.org, Freesound Radio \& Sons de Barcelona"). \emph{Facultat de Belles Arts (Faculty of Fine Arts)}, Universitat de Barcelona. Barcelona.\\
\years{2008a} \textsc{Panel Member} together with Alsina, A., de Jong, B., Loscos, A. and Roma, G. (September 27, 2008). “Influencia de la tecnología en la evolución de la música y la industria" (Influence of the technology in the evolution of music and industry). \emph{NetAudio}, CCCB. Barcelona. \href{https://www.youtube.com/watch?v=6JlCCvYXrHY}{[video]}\\
\years{2007}\textsc{Oral Presenter} together with Roma, G. (September 20, 2007). “A Sound Editor with a Tangible Interface". \emph{SCSymposium(2007)}, DCM, The Hague, The Netherlands.

\subsection*{Own Institution}
\noindent

\years{2018c} \textsc{Lightning Talk Speaker}. (September 28, 2018). “Challenges and Opportunities of Collaborative Music Live Coding: A Practitioner's Approach". \emph{The Raw and The Cooked, Inter/sections 2018}. Café 1001, London, UK.\\
\years{2018b} \textsc{Oral Presenter}. (August 13, 2018). “Women in Music Tech: A Case Study". \emph{Oppstartseminar (Institutt for musikk)}. Dokkhuset, Trondheim, Norway.\\
\years{2018a} \textsc{Oral Presenter}. (August 13, 2018). “A Journey Through My Research and Creative Practice". \emph{Oppstartseminar (Institutt for musikk)}. Dokkhuset, Trondheim, Norway.\\
\years{2017b} \textsc{Oral Presenter}. (June 19, 2017). “Computational Challenges of Co-creation in Collaborative Music Live Coding: An Outline". \emph{CCW2017: Co-Creation Workshop, ICC 2017}. Atlanta, GA, USA.\\
\years{2017a} \textsc{Panel Member} together with Ikkache, L. (May 4, 2017). “Women in Music Tech 2016--2017". Oral presentation and discussion. \emph{GTCMT}, Geogia Tech, Atlanta, GA, USA.\\
\years{2016d} \textsc{Lightning Talk Speaker}. (November 2, 2016). “Tangible User Interfaces and Tabletops". \emph{First Annual Women and Music Tech Concert and Reception}, The Garage, Atlanta, GA. USA.\\
\years{2016c} \textsc{Panel Member} together with Ikkache, L. and Jackson, D. (May 5, 2016). “Women in Sound." Oral presentation and discussion. \emph{GTCMT}, Geogia Tech, Atlanta, GA, USA.\\
\years{2016b} \textsc{Oral Presenter}. (February 25, 2016). “Algorithmic Composition: My Personal Journey". Oral presentation as a guest speaker in Jason Freeman's \emph{Computer Music Composition} class. GTCMT, Atlanta, GA, USA.\\
\years{2016a} \textsc{Oral Presenter}. (January 26, 2016). “EarSketch: Computational Music Remixing for All". Oral presentation as a guest speaker in Barbara Ericson's \emph{Educational Technology} class. College of Computing, Georgia Tech, Atlanta, GA, USA.\\
\years{2015c} \textsc{Oral Presenter}. (September 3, 2015). “Musical Tabletops: Challenges and Opportunities for Computer-Supported Collaborative Music and HCI". \emph{College of Architecture Research Forum}, Georgia Tech. Atlanta, GA, USA.\\ 
\years{2015b} \textsc{Oral Presenter}. (August 27, 2015). “Musical Tabletops: Challenges and Opportunities for Computer-Supported Collaborative Music and HCI". \emph{GVU Center Brown Bag Seminar Series}, Georgia Tech, Atlanta, GA, USA. \href{https://www.youtube.com/watch?v=VzOdtsq8YyE}{[video]}\\
\years{2015a} \textsc{Oral Presenter}. (August 24, 2015). “Musical Tabletops: Challenges and Opportunities for Computer-Supported Collaborative Music and HCI". \emph{GTCMT Seminar Series}, Georgia Tech, Atlanta, GA, USA.\\ 
\years{2014} \textsc{Oral Presenter}. (April 9, 2014). “Let's Jam the Reactable: Peer Learning During Musical Improvisation with a Tabletop Tangible Interface". \emph{London Knowledge Lab}, London.\\
\years{2013} \textsc{Oral Presenter}. (June 2, 2013). “Tabletop Groupware for Music Performance: Design and Evaluation". \emph{CRC PhD Student Conference 2013}, OU, Milton Keynes, UK.\\
\years{2012} \textsc{Oral Presenter}. (June 12, 2012). “Collaboration on Interactive Tabletops for Music Performance: An Exploratory Study". \emph{CRC PhD Student Conference 2012}, OU, Milton Keynes, UK.\\
\years{2011b} \textsc{Oral Presenter}. (June 16, 2011). “Tabletop Groupware for Music Performance: Design and Evaluation". \emph{CRC PhD Student Conference 2011}, OU, Milton Keynes, UK.\\
\years{2011a} \textsc{Oral Presenter}. (May 17, 2011). “Tabletop Groupware for Music Performance: Design and Evaluation". \emph{2011 Doctoral Workshops Conference}, OU, Milton Keynes, UK.\\
\years{2010b} \textsc{Oral Presenter}. (June 8, 2010). “Issues and Techniques for Collaborative Music Making on Multi-touch Surfaces". \emph{CRC PhD Student Conference 2010}, OU, Milton Keynes, UK.\\
\years{2010a} \textsc{Oral Presenter}. (May, 2010). “Issues and Techniques for Collaborative Music Making on Multi-touch Surfaces". \emph{Music Research Day}, Music Research Studio, OU, Milton Keynes, UK.

\section*{Poster Presentations, Demos \& Workshops}
\noindent

\subsection*{Poster Presentations \& Demos}
\noindent

\years{2018b} \textsc{Poster Presenter} together with Roma, G. (June 7, 2018). “Live Repurposing of Sounds: MIR Explorations with Personal and Crowdsourced Databases". \emph{NIME '18}. Blacksburg, Virginia, USA. \\
\years{2018a} \textsc{Demo Presenter} together with Skach, S. (March 19, 2018). “Embodied Interactions with E-Textiles and the Internet of Sounds for Performing Arts". \emph{TEI '18}. Stockholm, Sweden. \\
\years{2017c} \textsc{Poster Presenter}. (June 22, 2017). Authors: Weisling, A. and Xambó, A. “Constructing a Conceptual Framework for Collaborative Audiovisual Performance". \emph{ICCC '17}. Atlanta, GA, USA.\\
\years{2017b} \textsc{Poster Presenter}. (June 22, 2017). Authors: Weisling, A., Xambó, A., Magerko, B., Roma, G., Jacob, M., Bhanu, N., and Freeman, J. “TuneTable: A Tangible Computational Music Installation for Informal Learning". \emph{ICCC '17}. Atlanta, GA, USA.\\
\years{2017a} \textsc{Poster \& Demo Presenter}. (March 21, 2017). “Experience and Ownership with a Tangible Computational Music Installation for Informal Learning". \emph{TEI '17}. Yokohama, Japan.\\
\years{2016b} \textsc{Poster \& Demo Presenter}. (August 11, 2016). “Learning to Code Through MIR". \emph{Late-Breaking Demo Session of ISMIR 2016}. New York.\\
\years{2016a} \textsc{Poster \& Demo Presenter} together with Roma, G. (July 2, 2016). “Do the Buzzer Shake". \emph{ICLI 2016}. Brighton, UK.\\
\years{2015} \textsc{Poster \& Demo Presenter} together with McKlin, T. (August 14, 2015). “EarSketch: A STEAM Approach to Broadening Participation in Computer Science Principles". \emph{RESPECT 2015}. Charlotte, NC. USA.\\
\years{2014} \textsc{Poster Presenter} together with Price, S. (April 29, 2014). “Towards an Integrated Methodological Framework for Understanding Embodiment in HCI". \emph{CHI '14}. Toronto, ON. \href{https://www.youtube.com/watch?v=pRDHuCUltwo}{[video]}\\
\years{2012} \textsc{Demo Presenter}. (January 10, 2012). "Tangible Additive Sound Synthesis (TASS)". \emph{Welcome to the French Embassy}, OU. Milton Keynes, UK.\\
\years{2011d} \textsc{Poster Presenter}. (June 17, 2011). “Designing and Evaluating Interactive Systems: Musical Tabletops for Collective Music Performance". \emph{CRC PhD Student Conference 2011}, OU. Milton Keynes, UK.\\
\years{2011c} \textsc{Poster \& Demo Presenter} together with Milne, A. J. (May 30, 2011). “Hex Player — A Virtual Musical Controller". \emph{NIME '11}. Oslo, Norway.\\
\years{2011b} \textsc{Poster Presenter}. (March 8, 2011). “Designing and Evaluating Interactive Systems: Musical Tabletops for Collective Music Performance". \emph{The Open University Poster Competition 2011}. Milton Keynes, UK.\\
\years{2011a} \textsc{Poster Presenter}. (January 25, 2011) “TOUCHtr4ck: Democratic Collaborative Music". \emph{TEI '11}. Funchal, Madeira.\\
\years{2010} \textsc{Poster Presenter}. (June 8, 2010). “Issues and Techniques for Collaborative Music Making on Multi-touch Surfaces". \emph{CRC PhD Student Conference 2010}, OU. Milton Keynes, UK.\\
\years{2008b} \textsc{Poster Presenter}. (June 9--11, 2010). “Interfaces for Sketching Musical Compositions". \emph{SMC Summer School 2008}. Genoa, Italy.\\
\years{2008a} \textsc{Poster Presenter} together with Roma, G. (June 6, 2008). “A Tabletop Waveform Editor for Live Performance". \emph{NIME '08}. Genoa, Italy.


\subsection*{Workshops}
\noindent

\years{2018b} \textbf{Xambó, A.} (July 12--13, 2018). “Creative Audio Programming". \emph{Filmuniversität  Babelsberg Konrad Wolf}, Potsdam, Germany. Organized by MA Creative Technologies.\\
\years{2018a} Allik, A., \textbf{Xambó, A.} (April 7--8, 2018). “Collaborative Network Music". \emph{Rewire 2018}, The Hague, The Netherlands. Organized by Music Hackspace. Funded by Rewire.\\
\years{2017} \textbf{Xambó, A.} (October 14, 2017). “Huddersfield Girl Geeks: Audiovisual Creative Coding with P5.js". \emph{Kirklees Libraries}, Huddersfield, UK. Funded by Google.\\
\years{2013} \textbf{Xambó, A.} (May 2, 2013). “Introduction to SuperCollider". \emph{Music Computing Meeting}, OU. Milton Keynes, UK.\\
\years{2012} \textbf{Xambó, A.}; Roma, G. and Bovermann, T. (April 15, 2012). “Tangible Musical Interfaces with SuperCollider". \emph{SuperCollider Symposium 2012}, Goldsmiths, University of London. London.

\subsection*{Webinars}
\noindent

\years{2016} \textbf{Xambó, A.} (October 28, 2016). “Debugging with EarSketch". GTCMT, Georgia Tech, Atlanta, GA, USA.\\

\section*{Discography}
\noindent

\subsection*{Solo Albums}
\noindent

\years{2018}Anna Xambó. \emph{H2RI} [FLAC/MP3 files]. Chicago (IL, USA): pan y rosas.\\
\years{2013}peterMann. \emph{On the Go} [promo CD \& FLAC/MP3 files]. Barcelona: Carpal Tunnel.\\
\years{2011}peterMann. \emph{init} [promo CD \& FLAC/MP3 files]. Barcelona: Carpal Tunnel.

\subsection*{Band Albums}
\noindent

\years{1996}La Más Fina. \emph{Zande Phondex} [CD]. Barcelona: Apache Productions.\\
\years{1994}La Más Fina. \emph{Como quien dice la hoja iberia extrafina} [Cassette]. Barcelona: Self-released.\\
\years{1992}Sosa's Cáustica. \emph{Paraponera Clavata} [Cassette]. Barcelona: Murmur Town.

\subsection*{Participation in Compilations}
\noindent

\years{2018}peterMann. “n02-petermann" (11 min 10 sec). in \emph{Noiselets} [FLAC/MP3 files]. Barcelona: Carpal Tunnel.\\
\years{2016}peterMann. “Go wild y'all" (1 min). in \emph{Microtopies 2016} [MP3 files]. Barcelona: Gracia Territori Sonor.\\
\years{2015}peterMann. “ldnsktch01" (1 min). In \emph{Microtopies 2015} [MP3 files]. Barcelona: Gracia Territori Sonor.\\
\years{2010}peterMann. “init11" (3 min 29 sec). In \emph{Electronic music from Catalonia 2010} [CD]. Barcelona: Catalan! Arts / Sonar, Barcelona.

\subsection*{Broadcasting}
\noindent

\years{2018b}Anna Xambó's “H2RI.01-04". (June 21, 2018). Rare Frequency on WZBC 90.3 FM Newton Boston College Radio.\\
\years{2018a}Anna Xambó's “H2RI.07". (May 17, 2018). No Pigeonholes EXP on KOWS-FM.\\
\years{2013f}peterMann's “og02". (July 28, 2013). BiP\_HOp Generation on Radio Grenouille.\\
\years{2013e}peterMann's “og01", og05, og07 \& og09. (June 23, 2013). Framework radio \#426.\\
\years{2013d}peterMann's “og01". (March 28, 2013). Rare Frequency on WZBC 90.3 FM Newton Boston College Radio.\\
\years{2013c}peterMann's “og01" \& “og10". (March 2, 2013). Onda Sonora.\\
\years{2013b}peterMann's selection of \emph{On The Go}'s tracks. (February 3, 2013). RNE Atmósfera.\\
\years{2013a}peterMann's “og02". (February 2, 2013). Störung Radio 127 on ScannerFM.\\
\years{2010b}peterMann's “init 10--12". (December 18, 2010). Onda Sonora.\\
\years{2010a}peterMann's “init 2". (April 12, 2010). Sismógrafo.

\section*{Selected Performances}
\noindent

\subsection*{Solo Performances}
\noindent

\years{2008c}\textbf{peterMann.} (September 28, 2018). Live coding session. \emph{The Raw, Inter/sections 2018. Café 1001}. London, UK.\\
\years{2018b}\textbf{Xambó, A.} (September 19, 2018). Audience device participation piece. \textsc{Imaginary Berlin}. WAC '18. Factory Berlin. Berlin, Germany.\\  
\years{2018a}\textbf{Xambó, A.} (August 9, 2018). Live. \textsc{MareNostrum}. Cube Fest. Moss Arts Center. Blacksburg, VA, USA.\\
\years{2017}\textbf{peterMann}. (January 8, 2017). Live coding session. \emph{Noiselets: A Noise Music Microfestival}. Freedonia, Barcelona, Spain.\\
\years{2016b}\textbf{Xambó, A.} (April 22, 2016). Live coding with EarSketch. \emph{Women in Sound Women on Sound 2016: Educating girls in sound}. Jack Hylton Music Room, University of Lancaster. Lancaster, UK.\\
\years{2016a}\textbf{peterMann}. (April 22, 2016). Live. \emph{Women in Sound Women on Sound 2016: Educating girls in sound}. Jack Hylton Music Room, University of Lancaster. Lancaster, UK.\\
\years{2013}\textbf{Xambó, A.} (October 4, 2013). Live coding session. \emph{Perspectives on Multichannel Live Coding}. PHONOS. Sala Polivalent, UPF. Barcelona, Spain.\\
\years{2012}\textbf{peterMann.} (September 20, 2012). Live. \emph{Crispy Crunchy Creaky}. Niu. Barcelona, Spain.\\
\years{2006}\textbf{peterMann.} (June 10, 2006). Live. \emph{5a Mostra Sonora i Visual | Convent Sant Agustí}. Barcelona, Spain.

\subsection*{Collaborative Performances}
\noindent
%\years{2017e}Brown, N., Stolfi, A., Schroeder, F., Chudy, M., Pase, T., Wilkie, S., \textbf{Xambó, A.}, Ikkache, L., Weisling, A. (submitted). “Transmusicking II". \\
\years{2018b}Weisling, A., \textbf{Xambó, A.} (June 5, 2018). “Beckon". \emph{NIME '18}. Moss Arts Center: Anne and Ellen Fife Theatre. Blacksburg, VA, USA.\\
\years{2018a}Weisling, A., \textbf{Xambó, A.} (March 20, 2018). “Beacon". \emph{TEI '18}. Kulturhuset. Stockholm, Sweden.\\
\years{2017d}Brown, N., Chudy, M., Papadomanolaki, M., Wilkie, S., Pase, T., Stolfi, A., Schroeder, F., \textbf{Xambó, A.}, Ikkache, L., Freeman, J., Ganesh, S., Kerure, A., Narang, J., Tsuchiya, T. (August 25, 2017). “Transmusicking I". \emph{AM '17}. Oxford House Theatre. London, UK. \\
\years{2017c}\textbf{Xambó, A.}, Roma, G. (August 21, 2017). “Hyperconnected Action Painting". \emph{WAC 2017}. Oxford House Theatre. London, UK. \\
\years{2017b}Weisling, A., \textbf{Xambó, A.} (May 16, 2017). “Beacon". \emph{NIME 2017}. Stengade. Copenhagen, Denmark.\\
\years{2017a}Weisling, A., \textbf{Xambó, A.} (February 11, 2017). “Beacon". \emph{Root Signals Festival 2017}. Georgia Southern University. Statesboro, Georgia, United States.\\
\years{2012}pulso (Roma, G., \textbf{Xambó, A.}). (March 15, 2012). Live coding session. \emph{Live Coding Sessions}. Niu. Barcelona, Spain.\\
\years{2004}pulso (Roma, G., \textbf{Xambó, A.}). (May 29, 2004). Live. \emph{Minima Festival}. Gandía, Spain.\\
\years{2002}b4ng (Roma, G., \textbf{Xambó, A.}, Brugos, C., Clarens). (June 13, 2002). Live. \emph{Sonar Festival}. Barcelona, Spain. 

\section*{Mastering (other's work)}
\noindent

\years{2018}\emph{Noiselets} [FLAC/MP3 files]. Barcelona: Carpal Tunnel.

\section*{Other Creative Products}
\noindent

\subsection*{Awarded Music Hacks}
\noindent

\years{2014}“crowdj". \emph{Music Hack Day}. Barcelona, Spain.\\ 
Prize: Rdio prize.\\
Role: Concept, part of the implementation and user interface design.\\
Collaborator: Gerard Roma.\\
\years{2012b}“Soundscape Turntablism". \emph{Music Hack Day}. Barcelona, Spain.\\ 
Prize: Reactable prize, Zvooq prize.\\  
Role: Concept, part of the implementation and tangible user interface design.\\
Collaborator: Gerard Roma.\\
\years{2012a}“Soundscape DJ". \emph{Music Tech Fest}. London, UK.\\ 
Prize: Warp Records prize. \\
Role: Concept, part of the implementation and tangible user interface design.\\
Collaborator: Gerard Roma.

\subsection*{Code}
\noindent

\years{2018b}Embedded AudioCommons: \href{https://github.com/AudioCommons/embedded-audiocommons}{github.com/AudioCommons/embedded-audiocommons}.\\
Role: Concept and implementation.\\
\years{2018a}HCI Python Utils: \href{https://github.com/axambo/hci-python-utils}{github.com/axambo/hci-python-utils}.\\
Role: Concept and implementation.\\
\years{2017--present} WACastMix: \href{http://annaxambo.me/code/WACastMix}{annaxambo.me/code/WACastMix}.\\
Role: Concept and implementation.\\
\years{2016--present}MIRLC: \href{https://github.com/axambo/MIRLC}{github.com/axambo/MIRLC}.\\
Role: Concept and implementation.\\
\years{2017b} HAP: \href{https://github.com/axambo/HAP}{github.com/axambo/HAP}.\\
Role: Concept and implementation.\\
\years{2017a}Beacon: \href{https://github.com/axambo/beacon}{github.com/axambo/beacon}.\\
Role: Concept and implementation of the audio engine.\\
\years{2016}Algonoise.: \href{https://github.com/axambo/algonoise}{github.com/axambo/algonoise}.\\
Role: Concept and implementation.\\
\years{2014}SoundXY4: The Art of Noise: \href{https://github.com/axambo/soundxy4}{github.com/axambo/soundxy4}.\\
Role: Concept, implementation and tangible user interface design.\\
\years{2012}SoundXY: \href{https://github.com/axambo/soundxy2}{github.com/axambo/soundxy2}.\\
Role: Concept, implementation and tangible user interface design.

\subsection*{Video Creations \& Animation Films}
\noindent

\years{2003}Xambó, A. \emph{Cosmogonias} (3 min). Spain. Video creation | Animation film.\\
\years{2002b}Xambó, A. \emph{b.scope} (3 min). Spain. Video creation.\\ %3:23
\years{2002a}Xambó, A. \emph{Transdata Pr.} (5 min). Spain. Video creation.\\
\years{2000}Xambó, A.  \emph{clubsfera} (3 min). Spain. Video creation | Animation film.\\
\years{1999}Xambó, A. \emph{Mitösöma} (10 min). Spain. Video creation | Animation film.\\
\years{1998c}Xambó, A. \emph{Lufthansa} (3 min). Spain. Videoclip for La Más Fina.\\
\years{1998b}Xambó, A. \emph{Neila} (2 min). Spain. Video creation.\\
\years{1998a}Xambó, A. \emph{Sueños} (1 min). Spain. Video creation | Animation film.

\subsection*{Installations \& Visuals}
\noindent

\years{09/2002}\emph{I love Japan}, Circuit Festival, Barcelona.\\
%Description: Audiovisual installation for Divinas Palabras' Japan collection. \\
Role: Visuals.\\
Collaborators: Urtzi Grau (director), Emma Dünner, Jorge Meneses, Ana Otero.\\

\years{03/2002--08/2002}\emph{Astoria (cinema \& restaurant)}, Barcelona. \\
%Description: Visuals for the opening and first season of the Astoria (cinema \& restaurant) venue.\\
Role: Co-filming and visuals.\\ 
Collaborators: Babylon Cannes (concept).\\

\years{09/2001}\emph{Eme3density, Second Architectural Market}, Centre de Cultura Contemporània de Barcelona (CCCB), Barcelona.\\
Role: Visuals \& Flash programming. \\ %(Flash animation and programming of the website presented in the event).\\
Collaborators: Urtzi Grau (curator), Ana Otero (artistic director).

%\hrule
\section*{Teaching}
\noindent

\subsection*{Graduate Courses}
\noindent

\years{10/2018}Course: \emph{Human-Computer Interaction} (8 h). \# Students: $\sim$15. \\
Master of Music, Communication and Technology (MCT), Norwegian University of Science and Technology (NTNU), Trondheim, Norway.\\ 
Role: Creation of syllabus, creation of content, instruction and assessment.\\
\years{10/2018}Course: \emph{Physical Computing} (28 h). \# Students: $\sim$15. \\
Master of Music, Communication and Technology (MCT), Norwegian University of Science and Technology (NTNU), Trondheim, Norway.\\ 
Role: Creation of syllabus, creation of content, instruction and assessment.


\subsection*{Undergraduate Courses}
\noindent

\years{02/2004--06/2004}Course: \emph{Experimental Motion Graphics} (45 h). \# Students: $\sim$15. \\
Centre de la Imatge i la Technologia Multimèdia, Universitat Politècnica de Catalunya, Terrassa, Barcelona.\\ 
Role: Co-creation of syllabus, creation of content, instruction and assessment.\\
\years{10/2003--02/2004}Course: \emph{Crossmedia} (45 h). \# Students: $\sim$15. \\
BAU Escola de Disseny, Universitat de Vic, Barcelona.\\ 
Role: Co-creation of syllabus, creation of content, instruction and assessment.\\
\years{11/2003--06/2004}
Course: \emph{Digital Compositing with Adobe AfterEffects} (45 h). \# Students: $\sim$10. \\ 
Media Art Institute Fak d'Art, Barcelona.\\ 
Role: Creation of syllabus, creation of content, instruction and assessment.\\
\years{11/2003--06/2004}
Course: \emph{Photography in Motion} (45 h). \# Students: $\sim$10. \\ 
Media Art Institute Fak d'Art, Barcelona.\\ 
Role: Creation of syllabus, creation of content, instruction and assessment.\\
\years{11/2003--06/2004}
Course: \emph{Type in Motion} (45 h). \# Students: $\sim$10.\\
Media Art Institute Fak d'Art, Barcelona.\\ 
Role: Creation of syllabus, creation of content, instruction and assessment.\\
\years{11/1999--06/2003}
Course: \emph{Computer Animation} (90 h). \# Students: $\sim$15. \\ 
Media Art Institute Fak d'Art, Barcelona.\\ 
Role: Creation of syllabus, creation of content, instruction and assessment.

\subsection*{Professional Courses}
\noindent

\years{04/2004--05/2005}
Course: \emph{Usability} (12 h). \# Students: $\sim$5.\\ 
Crea Formación, Barcelona.\\
Role: Instruction.\\
\years{04/2004--05/2005}
Course: \emph{Internet Design Techniques} (12 h). \# Students: $\sim$5.\\ 
Crea Formación, Barcelona.\\
Role: Instruction.\\
\years{04/2004--05/2005}
Course: \emph{Web Design with DreamWeaver} (24 h). \# Students: $\sim$5.\\
Crea Formación, Barcelona.\\
Role: Instruction.\\ 
\years{04/2004--05/2005}
Course: \emph{Multimedia Content with Adobe Flash} (16 h). \# Students: $\sim$5. \\
Crea Formación, Barcelona.\\
Role: Instruction.\\
\years{04/2004--05/2005}
Course: \emph{Flash Programming} (20 h) \# Students: $\sim$5.\\ 
Crea Formación, Barcelona.\\
Role: Instruction.\\
\years{04/2004--05/2005}
Course: \emph{Theoretical Aspects in Graphic Design} (12 h). \# Students: $\sim$5.\\ 
Crea Formación, Barcelona.\\
Role: Instruction.\\
\years{04/2004--05/2005}
Course: \emph{Video Edition with Adobe Premiere} (60 h) \# Students: 1.\\
Crea Formación, Barcelona.\\
Role: Creation of syllabus, creation of content and instruction.

\subsection*{Preschool \& Primary School Courses}
\noindent

\years{03/2004--06/2004}
Course: \emph{Crossmedia infantil} (11 h). \# Students (6--7 years old): $\sim$8.\\
Escola Magòria, Barcelona.\\
Role: Co-creation of syllabus, creation of content, instruction and assessment.\\
\years{03/2004--05/2004}
Course: \emph{Crossmedia infantil} (9 h). \# Students (9--10 years old): $\sim$15.\\
Escola Costa i Llobera, Barcelona.\\ 
Role: Co-creation of syllabus, creation of content, instruction and assessment.\\
\years{03/2004--05/2004}
Course: \emph{Crossmedia infantil} (12 h). \# Students (3--4 years old): $\sim$8.\\
Escola Glòries, Barcelona.\\
Role: Co-creation of syllabus, creation of content, instruction and assessment.

\section*{Supervision}
\noindent

\years{11/2018--present}Appraisal Committee Member for Hilmar Thordarson, Norwegian Programme for Artistic Research, Norwegian Science and Technology University, Trondheim, Norway.\\
\years{01/2018--08/2018}Co-advisor of Tayjo Padmini Vaduru (master student in Computer Science, Queen Mary University of London) of her master project proposal on automated generation of soundscapes using content from Audio Commons. Master thesis title: ``Moodscape Generator: Automated Generation of Soundscapes''.\\
\years{05/2016--12/2017}Mentor and advisor the female graduate students of the student-led organization \emph{Women in Music Tech}, including the Chair of the organization, Léa Ikkache, the Editor-in-Chief of the newsletter Amruta Vidwans and female newcomers to the organization, such as Jyoti Narang.\\
\years{09/2015--05/2016}Co-advisor of Marc Huet and Travis Gasque (master's students in Digital Media, School of Literature, Media, and Communication) and Anna Weisling (PhD student in Digital Media, School of LMC) for their graduate design project TuneTable. This work has been part of Brian Magerko’s Digital Media studio course at Georgia Tech. From this work we have published at TEI '17 (see Peer-Reviewed Conference Papers) and we have informed a successful and competitive NSF-funded grant (Advancing Informal STEM Learning Grant).\\
\years{09/2015--05/2017}Co-advisor of Pratik Shah (master student in Human-Centered Computing, School of Interactive Computing) with the research and design on adding collaborative features to EarSketch, an online platform for learning code by making music. This work has been part of the design and development of the NSF-funded project EarSketch, led by Jason Freeman. From this work we have published at the conferences ICLI '16 and AM '17, and also at the AES journal in 2018 (see Peer-Reviewed Conference Papers).

\section*{Additional Experience}
\noindent

\subsection*{Concerts Co-Organization}
\noindent

\years{2017}“Noiselets: A Noise Music Microfestival". (January 8, 2017). Freedonia, Barcelona.\\
\years{2016c}“The First Annual Women in Music Tech: Concert and Reception". (November 2, 2016). The Garage. Atlanta, GA, USA.\\
\years{2016b}“Audience device participation". (April 5, 2016). \emph{Web Audio Conference 2016}, Georgia Tech. Atlanta, GA, USA.\\
\years{2016a}“Live coding and the audiovisual web". (April 4, 2016). \emph{Web Audio Conference 2016}, Georgia Tech. Atlanta, GA, USA.\\
\years{2013b}“Perspectives on multichannel live coding". (October 4, 2013). PHONOS. Sala Polivalent, UPF. Barcelona.\\
\years{2013a}“Live Coding Sessions II". (March 22, 2013). Niu. Barcelona.\\
\years{2012}“Live Coding Sessions". (March 15, 2012). Niu. Barcelona.

\subsection*{Blogging}
\noindent

\years{09/2018--present}\href{http://wonomute.no}{Women Nordic Music Technology}, the blog of the WoNoMute organization. Creator and Coordinator.\\ 
\years{08/2018--present}\href{https://mct-master.github.io}{MCT master blog}, the blog of the MCT master. Co-Creator and Coordinator.\\ 
\years{05/2017--present}\href{https://www.audiocommons.org}{Audio Commons}, the blog of the EU-funded project Audio Commons. Editor-in-Chief, Reviewer and Author.\\ 
\years{10/2016--present}\href{http://annaxambo.me/blog}{Anna Xambó's Blog}, the blog of my personal website. Creator and Author.\\ 
\years{05/2016--12/2017}\href{http://womeninmusictech.gatech.edu}{Women in Music Tech}, the newsletter of the Women in Music Tech organization. Co-Creator, Editor, Reviewer and Author.\\ 
\years{09/2013--08/2014}\href{http://midassblog.wordpress.com}{MIDAS's Blog}, the research blog of the MIDAS project. Co-Creator and Author. \\ 
\years{01/2010--12/2011} \href{http://postwimp.com}{postWIMP}, a blog on HCI and interaction design. Co-Creator and Author.\\
\years{03/2006--03/2009}\href{http://streetypes.blogspot.com}{streeTypes}, a blog on typography in public spaces. Creator and Author. 

\subsection*{Artistic Collective Projects}
\noindent

\years{2008--present}Co-Founder of the experimental electronic music label Carpal Tunnel. Barcelona.\\
\years{2002}Co-Founder and Member of b4ng, a multidisciplinary collective in search of new forms of audiovisual communication. Barcelona.\\
\years{1998--2000}Co-Founder and Member of the experimental video collective jesus13. Barcelona.

%\hrule
\section*{Professional Activities}
\noindent

\subsection*{Professional Organization Member}
\noindent
\emph{Association for Computing Machinery (ACM)}.\\
\emph{International Computer Music Association (ICMA)}.

\subsection*{Academic Chair / Committee Member / Conference Chair}
\noindent

\years{2019b}\textsc{Paper Co-Chair}. \emph{New Interfaces for Musical Expression 2019}. Porto Alegre, Brazil.\\ 
\years{2019a}\textsc{Programme Committee Member}. \emph{ACM Creativity \& Cognition 2019}. San Diego, USA.\\ 
\years{2018}\textsc{Programme Committee Member}. \emph{ACM Spatial User Interaction 2018}. Berlin, Germany.\\ 
\years{2017c}\textsc{Session Chair}. \emph{Web Audio Conference 2017}. London.\\
\years{2017b}\textsc{Programme Committee Member}. \emph{Second Conference on Computer Simulation of Musical Creativity}, Open University. Milton Keynes, UK.\\ 
\years{2017a}\textsc{Local Committee Member}. \emph{International Conference on Computational Creativity 2017}, Georgia Tech. Atlanta, GA, USA.\\
\years{2016b}\textsc{Co-Founder \& Co-Chair}. \emph{Women in Music Tech Committee}, GTCMT, Georgia Tech. Atlanta, GA, USA.\\
\years{2016a}\textsc{Music/Artworks Co-Chair}. \emph{Web Audio Conference 2016}, Georgia Tech. Atlanta, GA, USA.\\
\years{2011b}\textsc{Session Chair} (``Laptop/Coding/NI''). \emph{Internatio nal Computer Music Conference}. Huddersfield, UK.\\
\years{2011a}\textsc{Committee Member}. \emph{CRC PhD Student Conference 2011}, OU. Milton Keynes, UK.

\subsection*{Conference Reviewer}
\noindent

\years{2018i}\emph{ACM special interest group on Computer GRAPHics and Interactive Techniques (SIGGRAPH)} (2018).\\
\years{2018h}\emph{International Conference of the Learning Sciences} (2018).\\
\years{2018g}\emph{International Society for Music Information Retrieval Conference} (2018).\\
\years{2017h}\emph{ACM Creativity and Cognition} (2017).\\
\years{2017g}\emph{ACM Innovation and Technology in Computer Science Education} (2017).\\
\years{2017f}\emph{Co-Creation Workshop at International Conference on Computational Creativity} (2017).\\
\years{2017--2018}\emph{International Computer Music Conference -- ICMC Music} (2017e, 2018f).\\
\years{2016g}\emph{International Conference on Live Interfaces} (2016).\\
\years{2016f}\emph{ISSTA International Festival and Conference on Sound in the Arts, Science and Technology} (2016).\\
\years{2016--2018}\emph{Web Audio Conference} (2016e, 2017d, 2018e).\\
\years{2015--2017}\emph{ACM Special Interest Group on Computer-Human Interaction} (2015c, 2016d, 2017c, 2018d).\\
\years{2013c}\emph{IEEE Interactive Tabletops and Surfaces} (2013).\\
\years{2012--2018}\emph{ACM Designing Interactive Systems} (2012c, 2016c, 2018c).\\
\years{2012--2018}\emph{ACM Tangible, Embedded and Embodied Interaction} (2012b, 2013b, 2014b, 2015b, 2016b, 2017b, 2018b).\\
\years{2011--2018}\emph{New Interfaces for Musical Expression} (2011, 2012a, 2013a, 2014a, 2015a, 2016a, 2017a, 2018a).

\subsection*{Journal Reviewer}
\noindent

\years{2018d}\emph{PLOS One}. Public Library of Science.\\
\years{2018c}\emph{British Journal of Educational Technology}. Wiley.\\
\years{2018b}\emph{Journal of New Music Research}. ScholarOne Manuscripts.\\
\years{2018a}\emph{Transactions on Computing Education}. ACM.\\
\years{2017}\emph{Journal of Audio Engineering Society}. Audio Engineering Society.\\
\years{2016b}\emph{Interacting with Computers}. Oxford Journals.\\
\years{2016a}\emph{Qualitative Research}. Sage Publications.\\
\years{2015}\emph{International Journal of Human-Computer Studies}. Elsevier.

\subsection*{Panelist}
\noindent

%\years{2018d}(November 15, 2018). Resonate Music Conference 2018. Glasgow, Scotland.\\
\years{2018c}(October 26, 2018). \emph{Panel Session 3: Equality, Diversity, Gender} with Thomas Hilder (chair), Jill Diana Halstead Hjørnevik (panelist), Sunniva Skjøstad Hovde (panelist), Vivian Anette Lagesen (panelist), and Anna Xambó (panelist). Knowing Music -- Musical Knowing: Cross disciplinary dialogue on epistemologies. International Music Research School 2018, NTNU. Dokkhuset, Trondheim, Norway.\\
\years{2018b}(July 4, 2018) \emph{The Disturbing Discussion about Innovation} with Nicolas d'Alessandro (panelist), Tom Mitchell (panelist), Anna Xambó (panelist), and Matthias Strobel (moderator). Wallifornia MusicTech Hackathon. Liège, Belgium.\\
\years{2018a}(June 6, 2018) \emph{Future, Democratization, and Globalization of NIMEs} with Onyx Ashanti (panelist), Peter Nyboer (panelist), Anna Xambó (panelist), Pamela Z (panelist) and R. Benjamin Knapp (moderator). NIME '18. Moss Arts Center: Anne and Ellen Fife Theatre. Blacksburg, VA, USA.

\subsection*{Coach}
\noindent

\years{2018}(July 4, 2018) Wallifornia MusicTech Hackathon. Liège, Belgium.

\subsection*{Jury Member}
\noindent

\years{2018}\emph{COLLAB2018}, Institute of Electronic Music and Acoustics (IEM), University of Music and Performing Arts. Graz, Austria.\\
\years{2016}\emph{MOOG Hackathon 2016}, GTCMT, Georgia Tech. Atlanta, GA, USA.

\subsection*{Music Judge}
\noindent

\years{2018}\emph{Celebrating Women in Sound, 8 March 2018}, Goldsmiths University, London.\\
\years{2017b}\emph{National Student Electronic Music Event 2017}, Louisiana State University. Baton Rouge, LA, USA.\\
\years{2017a}\emph{EarSketch National Competition 2017}, GTCMT, Georgia Tech. Atlanta, GA, USA.

%\hrule
\subsection*{Consultancies}
\noindent

\years{08/2015--10/2015}\emph{Flux Project}, Atlanta, GA, USA.\\
Consulting on the development of interactive audio components of an art project for Flux Night 2015.\\
Collaborators: Jason Freeman (coordinator), Gerard Roma.

\subsection*{Entrepreneurship}
\noindent
\years{08/2018--present}\emph{Women Nordic Music Technology (WoNoMute)}, NTNU, Trondheim, Norway. \\
Co-Founder and Chair of \emph{WoNoMute}, an organization at NTNU that aims to promote and connect the work of women in music tech at local, national and international levels.\\\years{05/2016--12/2017}\emph{Women in Music Tech}, Atlanta, GA, USA. \\
Co-Founder and Co-Chair of \emph{Women in Music Tech}, the first student organization at GTCMT that looks into bringing more women into the program of music technology with actions on recruitment, external communication, internal communication, and creating a safe space.\\
\years{02/2004--06/2010}\emph{Nodular Soft}, Barcelona. \\
Co-Founder of a freelance studio focused on user-centric software and AV communication, development of community websites using several CMS, development of AV programs under specific needs, and usability consultancy.

\subsection*{Research Visits}

\years{07/2017}Filmuniversität Babelsberg KONRAD WOLF, Potsdam, Germany.\\
\years{05/2012}University of Strathclyde, Glasgow, Scotland, UK.\\
\years{06/2011}University of Strathclyde, Glasgow, Scotland, UK.\\
\years{04/2011--05/2011} UPF, Barcelona, Spain.\\
\years{03/2010--06/2010}The Open University, Milton Keynes, UK.

%\hrule
\section*{Skills}
\noindent

\subsection*{Languages}
\noindent

Catalan (native or bilingual proficiency), Spanish (native or bilingual proficiency), English (full professional proficiency), German (basic level), Italian (basic level), French (basic level).

\subsection*{Computer Skills}
\noindent

Operating Systems: OS X, Windows and Linux desktop (Ubuntu).\\
Programming: Actionscript, Assembly (basic level), C, CSS, Java, JavaScript, jQuery, MySQL, PHP, Python, Web Audio, XML.\\
Scientific Apps: MATLAB, Octave, R, SPSS.\\
Version Control Systems: CVS, Git, Subversion.\\
Music Apps: DAWs (Ableton Live, Cubase, Reaper, Logic Pro), Max/MSP, PureData, SuperCollider, wave editors (Audacity, SoundForge, WaveEditor).\\
Text \& Multimedia Analysis Apps: ELAN, MAXQDA, VCode.\\
Other Apps: Graphics and multimedia authoring apps (AfterEffects, Blender, Dreamweaver, Final Cut Pro, Flash, Freehand, Illustrator, InDesign, Photoshop, Premiere, Processing, Combustion, 3DMax), LaTeX, MS Office suite. CMS (Drupal, WordPress). Jekyll.\\
Hardware: Arduino, Bela.

%\vspace{1cm}
\vfill{}
%\hrulefill

\begin{center}
{\scriptsize  Anna Xambó •\- Curriculum Vitae •\- Last updated: \today\- •\- %original: Last updated: \today\- •\- 
% ---- PLEASE LEAVE THIS BACKLINK FOR ATTRIBUTION AS PER CC-LICENSE
Typeset in \href{http://nitens.org/taraborelli/cvtex}{
%\fontspec{Times New Roman}
\XeTeX }\\
% ---- FILL IN THE FULL URL TO YOUR CV HERE
\href{http://github.com/axambo/CV}{http://github.com/axambo/CV}}
\end{center}

\end{document}